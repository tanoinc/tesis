% Configuración
%   - Espacio entre el texto principal y las notas al pie
\setlength{\skip\footins}{1cm}
%   - Espacio entre notas al pie
\setlength{\footnotesep}{0.5cm}
%   - Formatos
% \setmainfont{Droid serif} % Si se desea cambiar la tipografía
\titleformat{\paragraph}{\normalfont\normalsize\bfseries}{\theparagraph}{1em}{}
\titlespacing*{\paragraph}{0pt}{3.25ex plus 1ex minus .2ex}{1.5ex plus .2ex}
\setlength{\parskip}{0.6em} % Separación entre párrafos
\setlength{\parindent}{2em} % Sangría de cada párrafo
%   - Resaltado de sintaxis
% \setminted{
%   fontsize=\small,
%   style=bw % o style=colorful para blanco y negro
% }

%     + Define el ambiente {phpcode} para código PHP
\newminted{php}{linenos,style=colorful}
%     + Define el comando \phpfile{path} para código PHP desde archivos externos
\newmintedfile{php}{linenos,style=colorful}
%     + Define el ambiente {bashcode} para comandos de CLI
% \newminted{bash}{breaklines=true}
%     + Define el comando \bashfile{path} para comandos de CLI desde archivos externos
% \newmintedfile{bash}{breaklines=true}
%     + Define el ambiente {jsoncode} para código JSON
\newminted{json}{}
%     + Define el comando \jsonfile{path} para código JSON desde archivos externos
% \newmintedfile{json}{}
%     + Define el ambiente {httpcode} para sesiones HTTP
\newminted{http}{breaklines=true,breakatwhitespace=false}
%     + Define el comando \httpfile{path} para sesiones HTTP desde archivos externos
\newmintedfile{http}{breaklines=true}
%   - Formato de los glosarios
\setglossarystyle{altlist}
%   - Formato de bloques preformateados
\lstset{
  basicstyle=\small\ttfamily,
  columns=flexible,
  breaklines=true
}
%   - Formato de los \subparagraphs para que tengan un salto de línea que los separe del texto
\titleformat{\subparagraph}{\normalfont\normalsize\bfseries}{\thesubparagraph}{1em}{}
\titlespacing*{\subparagraph}{\parindent}{3.25ex plus 1ex minus .2ex}{.75ex plus .1ex}
%   - Nombre y título para los bloques de código de minted
\renewcommand{\listingscaption}{Bloque de código}
\renewcommand{\listoflistingscaption}{Listado de bloques de código}

\captionsetup{aboveskip=0pt}