% Comandos personalizados

% {\fechaPresentacion} :: para escribir la fecha de presentación del trabajo
\newcommand{\fechaPresentacion}{\today}
% {\unlp} :: para escribir "Universidad Nacional de La Plata"
\newcommand{\unlp}{Universidad Nacional de La Plata}
% {\facultad} :: para escribir "Facultad de Informática"
\newcommand{\facultad}{Facultad de Informática}
% {\cespi} :: para escribir "CeSPI"
\newcommand{\cespi}{CeSPI}
% {\tituloTrabajo} :: Para escribir el título de la tesina
\newcommand{\tituloTrabajo}{Mi Universidad: Una aplicación móvil para mejorar la experiencia de usuario de los estudiantes de la Universidad Nacional de La Plata}
% \tituloTrabajoDosLineas :: Para escribir el título de la tesina en dos líneas (carátula)
\newcommand{\tituloTrabajoDosLineas}{Mi Universidad: Una aplicación móvil para mejorar la experiencia de usuario de los estudiantes de la Universidad Nacional de La Plata}
% {\miguelcarbone} :: para escribir "Luciano Coggiola"
\newcommand{\lucianoc}{Luciano Agustín Coggiola}


% \eng{English expression} :: para denotar que "English expression" está en inglés
\newcommand{\eng}[1]{\textit{#1}}

% {\caratula} :: para generar la carátula de la tesina
\newcommand{\caratula}{
  \begin{center}
    \huge{\unlp}\\
    \vspace{5mm}
    \large{\facultad}\\  
    \vspace{5mm}
    \includegraphics[width=40mm]{src/imagenes/caratula/escudo.png}\\
    \vspace{5mm}
    \huge{\tituloTrabajoDosLineas}\\
    \vspace{10mm}
    \large{Tesina de Licenciatura en Informática}\\
    \vspace{10mm}
    \large{\textbf{\lucianoc}}\\
    \vspace{15mm}
    \large{Director: Lic. Francisco Javier Díaz}\\
    \large{Codirectora: Lic. Ana Paola Amadeo}\\
    \large{Asesor Profesional: Lic. María Alejandra Osorio}\\
    \vspace{10mm}
    \normalsize{\fechaPresentacion}\\
  \end{center}
}

% {\checkmark} :: para imprimir un check (tilde)
\def\checkmark{\tikz\fill[scale=0.4](0,.35) -- (.25,0) -- (1,.7) -- (.25,.15) -- cycle;}

\newcommand{\figura}[4]{
	\begin{figure}[H]
	\begin{center}
	  \includegraphics[width=#4\linewidth]{src/imagenes/#1}
	  \caption{#2}
	  \label{fig:#3}
	  \end{center}
	\end{figure}
}