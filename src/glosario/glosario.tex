\newpage

\newglossaryentry{javascript} {
  name = {JavaScript},
  description = {JavaScript es un lenguaje interpretado y ligero. Comúnmente utilizado del lado del cliente en aplicaciones web (ejecutado en el browser). Existen también implementaciones \eng{server side}. },
}


\newglossaryentry{open source} {
  name = {Open Source},
  description = {\eng{Software} de código abierto},
}

\newglossaryentry{ajax} {
  name = {AJAX},
  description = {JavaScript asíncrono y XML, del inglés \eng{Asynchronous JavaScript And XML} },
}

\newglossaryentry{restful} {
  name = {RESTful},
  description = {Adjetiva a servicios o APIs, indicando que implementa REST},
}

\newglossaryentry{jquery} {
  name = {jQuery},
  description = {Librería \gls{javascript} que simplifica y potencia el desarrollo sobre \gls{javascript} y abstrae diferencias de implementación entre plataformas y navegadores Web.},
}

\newglossaryentry{bootstrap} {
  name = {Bootstrap},
  description = {Bootstrap es un \gls{framework} \gls{css} \gls{open source} para diseño de interfaces Web. },
}

\newglossaryentry{clustering} {
  name = {Clustering},
  description = {Agrupamiento de computadoras conectadas por una red de alta velocidad que figuran como si fuesen una sola computadora},
}

\newglossaryentry{github} {
  name = {GitHub},
  description = {Plataforma de desarrollo colaborativo basada en Git.},
}

\newglossaryentry{stackoverflow} {
  name = {Stack Overflow},
  description = {Sitio web para la comunidad informática de desarrolladores en el cual se pueden hacer preguntas y responder sobre cuestiones de desarrollo.},
}

\newglossaryentry{crud} {
  name = {CRUD},
  description = {Sigla en inglés de \eng{Create Read Udate Delete} que significan el listado de operaciones básicas asociadas a una entidad: Crear, leer, actualizar y eliminar.},
}

\newglossaryentry{ios} {
  name = {iOS},
  description = {Sistema operativo desarrollado por Apple Inc. para dispositivos
  como smartphones y tabletas}, sort = {ios}
}

\newglossaryentry{android} {
  name = {Android},
  description = {Sistema operativo desarrollado por Google para dispositivos
  como smartphones y tabletas},
  sort = {android}
}

\newglossaryentry{smartphone} {
  name = {smartphone},
  description = {Nombre en inglés dado a los teléfonos inteligientes},
  plural= {smartphones},
  sort = {smartphone}
}

\newglossaryentry{metodo_get} {
  name = {GET},
  description = {Método HTTP para obtener un recurso},
}

\newglossaryentry{WSDL} {
  name = {WSDL},
  description = {\eng{Web Services Description Language}, es un formato del Extensible Markup Language (XML) que se utiliza para describir servicios web (WS)},
  sort = {WSDL}
}

\newglossaryentry{framework} {
  name = {Framework},
  plural= {Frameworks},
  description = {Es un entorno de trabajo: representa un conjunto de herramientas enfocadas a resolver problemáticas comunes evitando reimplementarlas en cada nuevo desarrollo.}
}

\newglossaryentry{microframework} {
  name = {Microframework},
  description = {Es un \gls{framework} con funcionalidades innecesarias removidas para que sea más liviano.}
}


\newglossaryentry{laravel} {
  name = {Laravel},
  description = {Es un \gls{framework} \gls{php} (que corre del lado del servidor) para desarrollar aplicaciones web.}
}

\newglossaryentry{hhvm} {
  name = {HHVM},
  description = {\eng{HipHop Virtual Machine} es una máquina virtual desarrollada por \textit{Facebook}, \eng{open source}, que optimiza la ejecución de código \gls{php} y Hack.}
}

\newglossaryentry{markdown} {
  name = {Markdown},
  description = {Es un lenguaje de marcado ligero con una sintaxis de formato expresada en texto plano}
}

\newglossaryentry{nginx} {
  name = {Nginx},
  description = {Servidor web liviano de alto rendimiento.}
}

\newglossaryentry{phpfpm} {
  name = {PHP-FPM},
  description = {\eng{FastCGI Process Manager} de \gls{php} es una implementación de FastCGI optimizada para la ejecución de \gls{php} en sitios de alto tráfico.}
}

\newglossaryentry{symfony} {
  name = {Symfony},
  description = {Symfony es un \eng{framework} \gls{php} basado en MVC para el desarrollo de aplicaciones Web.}
}

\newglossaryentry{chulupi} {
  name = {Chulupí},
  description = {\gls{framework} desarrollado por el SIU para la implementación de Guaraní Web.}
}

\newglossaryentry{guarani} {
  name = {Guaraní},
  description = {Sistema desarrollado por el SIU para la gestión académica de alumnos desde su ingreso hasta su egreso.}
}

\newglossaryentry{XML} {
  name = {XML},
  description = {Del inglés \eng{eXtensible Markup Language}, que significa Lenguaje de Marcas Extensible: es un meta-lenguaje que permite definir lenguajes de marcas.},
  sort = {XML}
}
\newglossaryentry{phonegap} {
  name = {PhoneGap},
  description = {\gls{framework} (propiedad de \textit{Adobe}) para el desarrollo de aplicaciones móviles híbridas, basado en \gls{javascript} y \gls{html}5.}
}

\newglossaryentry{typescript} {
  name = {TypeScript},
  description = {Lenguaje \gls{open source} desarrollado por Microsoft que extiende la sintaxis de \gls{javascript} para añadir tipado estático y objetos basados en clases. Este código se \comillas{transpila} a código \gls{javascript} y es totalmente compatible con este.}
}

\newglossaryentry{replay attack} {
  name = {Replay Attack},
  description = {Ataque de reproducción en español: es un ataque de red y consiste en que una transmisión de datos válida sea fraudulentamente repetida.}
}


\newacronym{mime}{MIME}{\eng{Multipurpose Internet Mail Extensions}}
\newacronym[plural=URIs]{uri}{URI}{\eng{Uniform Resource Identifier}}
\newacronym[plural=Apps]{app}{App}{Aplicación móvil}
\newacronym{json}{JSON}{\eng{JavaScript Object Notation}}
\newacronym{http}{HTTP}{\eng{HyperText Transfer Protocol}}
\newacronym{w3c}{W3C}{\eng{World Wide Web Consortium}}
\newacronym{ietf}{IETF}{\eng{Internet Engineering Task Force}}
\newacronym{uddi}{UDDI}{\eng{Universal Description, Discovery and Integration}}
\newacronym[plural=APIs]{api}{API}{\eng{Application Programming Interface}}
\newacronym{rest}{REST}{\eng{REpresentational State Transfer}}
\newacronym{soap}{SOAP}{\eng{Simple Object Access Protocol}}
\newacronym{smtp}{SMTP}{\eng{Simple Mail Transfer Protocol}}
\newacronym{hmac}{HMAC}{Código de autenticación de mensajes en clave/\eng{hash}}
\newacronym{php}{PHP}{PHP: \eng{Hypertext Preprocessor}}
\newacronym{html}{HTML}{\eng{HyperText Markup Language}}
\newacronym{https}{HTTPS}{\eng{HyperText Transfer Protocol Secure}}
\newacronym{foss}{FOSS}{\eng{Free and open-source software}}
\newacronym{gnugpl}{GNU GPL}{\eng{GNU General Public License}}
\newacronym{cespi}{CeSPI}{Centro Superior para el Procesamiento de la Información}
\newacronym{siu}{SIU}{Sistema de Información Universitaria}
\newacronym{url}{URL}{\eng{Uniform Resource Locator}}
\newacronym{mvc}{MVC}{\eng{Model View Controller}}
\newacronym{gps}{GPS}{Sistema de Posicionamiento Global}
\newacronym{sdk}{SDK}{\eng{Software Development Kit}}
\newacronym{css}{CSS}{\eng{Cascading Style Sheets}}
\newacronym{pwa}{PWA}{\eng{Progressive Web Apps}}
\newacronym{i18n}{i18n}{Internacionalización}
\newacronym{jwt}{JWT}{\eng{JSON Web Token}}
