\newpage

\newglossaryentry{db:nosql} {
  name = {NoSQL},
  description = {nueva tendencia en motores de bases de datos que están orientadas a mejorar problemas posibles del paradigma tradicional de las bases de datos relacionales. Como principio, son no relacionales, distribuidas y escalables horizontalmente. Algunos ejemplos son Hadoop, Cassandra, CouchDB y MongoDB},
  sort = {nosql}
}

\newglossaryentry{ios} {
  name = {iOS},
  description = {Sistema operativo desarrollado por Apple Inc. para dispositivos
  como smartphones y tabletas}, sort = {ios}
}

\newglossaryentry{smartphone} {
  name = {smartphone},
  description = {Nombre en inglés dado a los teléfonos inteligientes},
  plural= {smartphones},
  sort = {smartphone}
}


\newacronym{acro:api}{API}{\eng{Application Programming Interface}}
\newacronym{lang:html}{HTML}{\eng{HyperText Markup Language}}
\newacronym{proto:http}{HTTP}{\eng{HyperText Transfer Protocol}}
\newacronym{proto:https}{HTTPS}{\eng{HyperText Transfer Protocol Secure}}
\newacronym{lang:json}{JSON}{\eng{JavaScript Object Notation}}
\newacronym{acro:rest}{REST}{\eng{REpresentational State Transfer}}
\newacronym{acro:soa}{SOA}{\eng{Service-Oriented Architecture}}
\newacronym{acro:url}{URL}{\eng{Uniform Resource Locator}}
\newacronym{acro:urn}{URN}{\eng{Uniform Resource Name}}
\newacronym{acro:uddi}{UDDI}{\eng{Universal Description Discovery and Integration}}
