\newpage

\newglossaryentry{ios} {
  name = {iOS},
  description = {Sistema operativo desarrollado por Apple Inc. para dispositivos
  como smartphones y tabletas}, sort = {ios}
}

\newglossaryentry{smartphone} {
  name = {smartphone},
  description = {Nombre en inglés dado a los teléfonos inteligientes},
  plural= {smartphones},
  sort = {smartphone}
}

\newglossaryentry{WSDL} {
  name = {WSDL},
  description = {Web Services Description Language, es un formato del Extensible Markup Language (XML) que se utiliza para describir servicios web (WS)},
  sort = {WSDL}
}

\newglossaryentry{XML} {
  name = {XML},
  description = {Del inglés \eng{eXtensible Markup Language}, que significa Lenguaje de Marcas Extensible: es un meta-lenguaje que permite definir lenguajes de marcas.},
  sort = {XML}
}
\newacronym{mime}{MIME}{\eng{Multipurpose Internet Mail Extensions}}
\newacronym{uri}{URI}{\eng{Uniform Resource Identifier}}
\newacronym{json}{JSON}{\eng{JavaScript Object Notation}}
\newacronym{http}{HTTP}{\eng{HyperText Transfer Protocol}}
\newacronym{w3c}{W3C}{\eng{World Wide Web Consortium}}
\newacronym{ietf}{IETF}{\eng{Internet Engineering Task Force}}

\newacronym{acro:api}{API}{\eng{Application Programming Interface}}
\newacronym{lang:html}{HTML}{\eng{HyperText Markup Language}}
\newacronym{proto:https}{HTTPS}{\eng{HyperText Transfer Protocol Secure}}
\newacronym{acro:rest}{REST}{\eng{REpresentational State Transfer}}
\newacronym{acro:soa}{SOA}{\eng{Service-Oriented Architecture}}
\newacronym{url}{URL}{\eng{Uniform Resource Locator}}
\newacronym{acro:urn}{URN}{\eng{Uniform Resource Name}}

