\subsection{Objetivo}
\label{objetivo}

Esta tesina se centra en el desarrollo de una aplicación móvil que permita
integrar múltiples servicios de la Universidad Nacional de La Plata (en
particular para este proyecto: SIU Guaraní y Moodle), con el objetivo de mejorar
la experiencia de usuario de los estudiantes, potenciando las posibilidades de
comunicación y colaboración entre ellos, la Universidad y sus dependencias, la
sociabilización de contenidos y la presencia de la Universidad.

De ello se desprenden los siguientes objetivos específicos:

\begin{itemize}
  \item Analizar la aplicaciones universitarias existentes en Argentina y el mundo para determinar las características más relevantes de este desarrollo.

  \item Analizar las herramientas existentes para llevar a cabo el desarrollo
  de una aplicación multiplataforma sobre dispositivos móviles, aprovechando las
  posibilidades tecnológicas que estos poseen.

  \item Analizar la integración de la aplicación móvil con los distintos
  servicios Web que brinda la Universidad.

  \item Desarrollar una aplicación móvil que permita comunicar y representar la
  información obtenida de los distintos servicios.

  \item Definir (y desarrollar) una interfaz de comunicación (\gls{api}) genérica y segura, que permita integrar los servicios.

  \item Desarrollar la integración de la aplicación móvil con SIU Guaraní,
  contribuyendo a la comunidad de desarrolladores que implementan este sistema en todas las Universidades Nacionales.

  \item Desarrollar la integración con Moodle aportando la posibilidad de
  conexión para quienes utilicen este sistema \eng{open source}.
\end{itemize}

Se considera que estos objetivos no deben estar orientados solo a la \unlp{}, sino que también contemplen a otras Universidades Nacionales. Es por ello que, además de requerimientos de diseño, se prioriza el código abierto como un aspecto fundamental en las decisiones a tomar.

Este desarrollo se implementa utilizando la arquitectura cliente-servidor, por lo que consta de dos partes:
\begin{itemize}
	\item Del lado del servidor: una \gls{api} REST escalable implementada en PHP utilizando JSON como formato de representación de los datos y OAuth como protocolo de autorización. Dicha \gls{api} permitirá la comunicación con los clientes móviles y los servicios de la Universidad.
	\item Del lado del cliente: una aplicación móvil multiplataforma implementada para Android que se comunicará con la \gls{api} mencionada anteriormente. Utilizará la tecnología del teléfono para llevar a cabo las nuevas operaciones de notificación y acceso ala información.
\end{itemize}

Inicialmente, también será necesario implementar algunos servicios que interactúen con dicha \gls{api}. Para ello, se desarrollara la integración con:
\begin{itemize}
	\item SIU \gls{guarani}: Se va a implementar un conector utilizando el framework \gls{siu} \gls{chulupi} (creado por el \gls{siu} y utilizado en \gls{guarani} Web, versión 2.9 y 3) que permita comunicar e integrar los recursos de Guaraní.
	\item Moodle: Se implementa un conector como plugin de Moodle que permite comunicar las novedades y acceder a otros recursos.
\end{itemize}

Con la finalización de la Tesina se pretende desarrollar el servicio completo de una aplicación móvil para los alumnos de la Universidad Nacional de La Plata, para constibuir a la mejora en la comunicación con sus estudiantes, su sociabilización e integración de sus servicios Web.

Como análisis posterior, se aplican técnicas de usabilidad para verificar el impacto en el uso de la herramienta. Estas permiten obtener un \eng{feedback} de los potenciales usuarios.
