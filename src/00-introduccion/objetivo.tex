\subsection{Objetivo}
\label{objetivo}

Esta tesina se centra en el desarrollo de una aplicación móvil que permita
integrar múltiples servicios de la Universidad Nacional de La Plata (en
particular para esta Tesina: SIU Guaraní y Moodle), con el objetivo de mejorar
la experiencia de usuario de los estudiantes, potenciando las posibilidades de
comunicación y colaboración entre ellos, la Universidad y sus dependencias, la
sociabilización de contenidos y la presencia de la Universidad.

De ello se desprenden los siguientes objetivos específicos:

\begin{itemize}
  \item Analizar las herramientas existentes para llevar a cabo el desarrollo
  de una aplicación multiplataforma sobre dispositivos móviles, aprovechando las
  posibilidades tecnológicas que estos poseen.

  \item Analizar aplicaciones universitarias existentes en Argentina y el mundo.

  \item Analizar la integración de la aplicación móvil con los distintos
  servicios Web que brinda la Universidad.

  \item Desarrollar una aplicación móvil que permita comunicar y representar la
  información obtenida de los distintos servicios.

  \item Desarrollar la integración de la aplicación móvil con SIU Guaraní,
  contribuyendo a la comunidad de desarrolladores que implementan este sistema en todas las Universidades Nacionales.

  \item Desarrollar la integración con Moodle aportando la posibilidad de
  conexión para quienes utilicen este sistema open source.
\end{itemize}
