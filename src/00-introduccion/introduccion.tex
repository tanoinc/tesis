El avance tecnológico de los dispositivos móviles ha ido creciendo rápidamente en los últimos años. Cada año nace una nueva generación de smartphones y tabletas incorporando nuevas tecnologías (GPS, lector de huellas, notificaciones, acelerómetro, cámara, etcétera). Estas permiten obtener mayor información para adicionar nuevas funcionalidades, mejorar la experiencia de usuario y potenciar la comunicación. De este modo, las aplicaciones móviles (apps) permiten establecer un buen canal de transmisión entre una entidad y el usuario o entre los distintos usuarios (que comparten un interés en común). 

Para este trabajo, referiré como “entidad” a la Universidad Nacional de La Plata (UNLP) y sus Facultades, las cuales en sus sistemas albergan una gran cantidad de información sobre las personas que allí desarrollan sus actividades. Dentro de ese conjunto de datos figuran los pertenecientes a los alumnos, población universitaria a la que apuntaré y voy a considerar “usuarios”.

Tomando como punto de partida el hecho de que los dispositivos inteligentes se han convertido en un accesorio indispensable para la vida cotidiana, y su aceptación es socialmente masiva, considero  que el desarrollo de una aplicación que comunique a la “entidad” con los “usuarios” será útil para mejorar, por parte de los alumnos, el acceso a la información de los servicios de la Universidad, así como también fomentará su presencia.

Además, al utilizarse sistemas abiertos e implementados por todas las Universidades de la Argentina, esta solución podrá expandirse hacia ellas.

Por lo anteriormente expuesto se hace evidente la necesidad de apropiación de las tecnologías arriba mencionadas, para el desarrollo de una aplicación que mejore la comunicación usuario-usuario y usuario-entidad.
