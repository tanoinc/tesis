El avance tecnológico de los dispositivos móviles ha ido creciendo rápidamente
en los últimos años. Cada año nace una nueva generación de smartphones y
tabletas incorporando nuevas tecnologías (GPS, lector de huellas,
notificaciones, acelerómetro, cámara, etcétera). Estas permiten obtener mayor
información para adicionar nuevas funcionalidades, mejorar la experiencia de
usuario y potenciar la comunicación. De este modo, las aplicaciones móviles
(apps) permiten establecer un buen canal de transmisión entre una entidad y el
usuario o entre los distintos usuarios (que comparten un interés en común).

Los dispositivos móviles son formadores de hábito en sus usuarios: existe una tendencia a revisar repetitivamente por períodos cortos, contenido dinámico, fácilmente accesible desde el celular\cite{oulasvirta2012habits}.
Estos hábitos motivan al usuario a realizar otras tareas con el teléfono, incrementando el tiempo total de su uso.

Para este trabajo, referiré como “entidad” a la Universidad Nacional de La Plata
(UNLP), sus dependencias y subdivisiones, dentro de las cuales en sus sistemas guardan una gran cantidad
de información sobre las personas que allí desarrollan sus actividades. Dentro
de ese conjunto de datos figuran los pertenecientes a los alumnos, población
universitaria a la que apuntaré y voy a considerar “usuarios”.

Según un estudio realizado por Google sobre el uso de \eng{\glspl{smartphone}}
en Argentina, \cite{GoogleEstudioSmartphones} ``la penetración de los teléfonos
inteligentes actualmente alcanza al 24\% de la población, y sus propietarios
dependen cada vez más de sus dispositivos. El 71\% de estos usuarios accede a
Internet todos los días desde su teléfono inteligente, y casi nunca sale de su
casa sin llevarlo.''. Se hace evidente el hecho de que los dispositivos
inteligentes se han convertido en un accesorio indispensable para la vida
cotidiana, y su aceptación es socialmente masiva, por lo que considero que el
desarrollo de una aplicación que comunique a la “entidad” con los “usuarios”
será útil para mejorar la experiencia de usuario, por parte de los alumnos, el
acceso a la información de los servicios de la Universidad, así como también
fomentará la presencia de la misma.

Además, al utilizarse sistemas abiertos e implementados por todas las
Universidades de la Argentina, esta solución podrá expandirse hacia ellas.

Por lo expuesto se hace evidente la necesidad de uso de
las tecnologías arriba mencionadas, para el desarrollo de una aplicación que
enriquezca la comunicación entre los alumnos y la entidad para mejorar su
experiencia de usuario.

Para llevar esto a cabo también es necesario proveer de una interfaz de comunicación genérica que permita integrar los servicios disponibles en la Universidad. El resultado de esta integración facilita a los usuarios el acceso a la información y brinda la posibilidad de utilizar características propias de los dispositivos móviles a sistemas externos. También les permite tener acceso al mundo de los \eng{smartphones} sin requerir el desarrollo completo de una aplicación móvil y agrupando la información en un solo lugar.
