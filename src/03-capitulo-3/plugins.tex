\subsection{Plugins}
\label{desarrollo_plugins}

Como parte de este proyecto, además del desarrollo de la aplicación móvil (y su \eng{backend}), los objetivos son desarrollar extensiones para sistemas utilizados en la \unlp{}. En particular se utilizan servicios en los que tengo acceso debido a mi trabajo en el \gls{cespi}.

A continuación, se pasan a detallar cuestiones relacionadas con la integración de: SIU Guaraní y Moodle.

\subsubsection{Personalización SIU Guaraní}
\label{desarrollo_plugins_guarani}

En Guaraní se utiliza el esquema de personalizaciones de \gls{chulupi} para desarrollar la integración con \nombreApp{}. 

El significado de los contextos está dado por las materias. Esto significa que un usuario de la aplicación móvil podría suscribirse a la materia que está interesado.

La personalización se denomina \comillas{mi\_universidad} e implementa (en principio) tres funcionalidades:
\begin{itemize}
\item Interconexión entre usuarios. Esta operación es utilizada cuando un usuario añade un servicio de Guaraní. Esta envía a \nombreApp{} el identificador externo del usuario \textit{logueado}.
\item Envío de mensajes. Al enviar mensajes en Guaraní, se enviarán novedades a \nombreApp{}. El alcance de las mismas estará determinado por los destinatarios del mensaje. Si este es a toda una comisión o mesa de examen, se enviará a todos los inscriptos y además tendrá la materia como contexto asociado. Esto significa que todos los interesados (más allá de los inscriptos) podrán ver en sus noticias el mensaje.
\item Fechas de parciales. Guaraní permite, desde su módulo \textit{Docente}, definir los parciales en cada comisión. La personalización envía a \nombreApp{} la fecha de los mismos, asociados al contexto de la materia y a todos sus inscriptos. 
\end{itemize}

\figura{03-capitulo-3/plugin_guarani_envio_mensaje.png}{Guaraní: Pantalla de envío de mensajes que se verán en la \eng{app}}{plugin_guarani_envio_mensaje}{0.7}

Se planifica incorporar más novedades y eventos del calendario, como por ejemplo, fechas de inscripción a cursadas, mesas de examen, horarios y aulas de materias y notificación de carga de notas, entre otras.

Esta personalización estará disponible para otras Universidades que deseen implementar \nombreApp{}.

\subsubsection{Plugin Moodle}
\label{desarrollo_plugins_moodle}

Otra de las extensiones propuestas en los objetivos de esta tesina es la de Moodle. Se trabajó con la última versión 3.3.1 pudiendo ser portada a versiones anteriores.

El tipo de extensión de Moodle utilizada es la de \textit{messaging consumers}\footnote{Se puede ver en \url{https://docs.moodle.org/dev/Messaging_consumers}}. Esta permite conectar un componente a ciertos tipos de eventos, y es configurable desde Moodle por el administrador y personalizable por los usuarios.

Para Moodle se define que los contextos serán los cursos.

El \textit{plugin} implementa, en principio, dos funcionalidades:
\begin{itemize}
\item Mensajería: Los mensajes personales que se reciban dentro de la plataforma, se envían como novedades personales con notificación a \nombreApp{}. De esta manera un alumno podrá saber con una notificación cuando le envían un mensaje privado. Esta característica se puede deshabilitar y es configurable por el usuario desde Moodle.
\item Foro de novedades y consultas: Los anuncios que se realicen en el foro de novedades, serán enviados a través de \nombreApp{} a los alumnos del curso y además quedarán asociados a su contexto. De esta manera los alumnos que no pertenezcan al curso pero quieran estar al tanto de las novedades, puedan hacerlo, suscribiéndose.
\end{itemize}

\figura{03-capitulo-3/plugin_moodle_config.png}{Moodle: Pantalla de configuración de parámetros de \nombreApp{}.}{plugin_moodle_config}{0.7}
\figura{03-capitulo-3/plugin_moodle_config_notificaciones.png}{Moodle: Pantalla de configuración de qué notificará la aplicación}{plugin_moodle_config_notificaciones}{0.7}

La extensión queda disponible con licencia \gls{gnugpl} versión 3 en \url{https://github.com/tanoinc/moodle-message_miuniversidad}.