\subsection{Pruebas de Usabilidad}
\label{desarrollo_pruebas_usabilidad}

Se hicieron pruebas de usabilidad sobre la aplicación móvil en base a las recomendaciones de estudio de \gls{nng}\cite{nng2017usabilityTest}\footnote{El grupo es liderado por Jackob Nielsen y Don Norman}. Para ello se orientó el estudio a obtener datos de una investigación cualitativa (con participación de usuarios) y a resolver cuestiones relacionadas con el diseño. 
Dentro de estas pruebas, se destacan por su popularidad y sus múltiples
áreas de aplicación, los llamados \eng{focus group} (o \comillas{grupos de discusión}).
Estos son una forma de entrevista grupal que se centra en la comunicación entre los participantes de la investigación para generar nueva información (datos, ideas, correcciones, etcétera)\cite{kitzinger1995qualitative}.

Se realizaron dos reuniones con grupos de usuarios\footnote{Coordinadas y guiadas con la ayuda Ivana Harari y Paola Amadeo}: el 18 de agosto y 13 de septiembre de 2017. A cada participante individualmente se le asignó una lista de tareas y a medida que se realizaban se tomaba nota de las dificultades, ideas y aclaraciones que surgían. La charla grupal favoreció a la prueba. La interacción entre los participantes permitió explorar el conocimiento y experiencia de las personas. De esta manera se pudo explorar no sólo lo que pensaban, sino también cómo lo hacían y por qué.

Los encuentros se llevaron a cabo en el \gls{cespi}, fueron en persona y moderados. Los participantes pertenecían a diversos ámbitos relacionados con la Universidad (alumnos, egresados, docentes y no docentes).

\subsubsection{Etapa preliminar}
\label{desarrollo_pruebas_usabilidad_preparativos}

Esta etapa consistió el la preparación de los elementos necesarios para las reuniones. Estos son indicados en las siguientes secciones.

\paragraph{Preparación del producto}
\label{desarrollo_pruebas_usabilidad_preparativos_producto}

Se configuró un servidor de prueba con la \gls{api}, apuntando a servicios externos de desarrollo (\gls{guarani} y Moodle). Además se crearon aplicaciones de \eng{testing} en Facebook (para relacionar las cuentas de usuario), en Google (para los mapas) y en los servicios de Ionic (para enviar notificaciones).

Para las pruebas de ingreso con Facebook fue necesario agregar a las personas como participantes de la aplicación de prueba (en Facebook). Esto es necesario ya que la utilizada para el \eng{test} no es pública.

\paragraph{Selección de participantes}
\label{desarrollo_pruebas_usabilidad_preparativos_participantes}

El grupo de personas y sus perfiles consistió en las indicados en el cuadro \ref{perfiles_usabilidad}.

\begin{table}[htbp]
\centering
\caption{Perfiles de usuarios para pruebas de Usabilidad}
\label{perfiles_usabilidad}
\begin{tabular}{|l|l|p{2.5cm}|p{2.5cm}|l|l|}
\hline
   & \multicolumn{1}{c|}{\textbf{Nombre}} & \multicolumn{1}{c|}{\textbf{Ocupación}} & \multicolumn{1}{c|}{\textbf{Estudios}} & \textbf{Edad} & \textbf{Egresado} \\ \hline
U1 & Tomás                                & Estudiante, Empleado                    & Periodismo (UNLP)                      & 29                    & No                \\ \hline
U2 & Maximiliano\footnotemark{}                          & Estudiante                              & Informática (UNLP), Locución           & 31                    & No                \\ \hline
U3 & Francisco                                     & Estudiante                              & Ingeniería/Informática (UNLP)          & 22                    & No                \\ \hline
U4 & Graciela                             & Empleada                                & Informática (UNLP)                     & 51                    & Si                \\ \hline
U5 & Quimey                               & Docente (UNLP)                                 & Bellas Artes (UNLP)                    & 28                    & Si                \\ \hline
\end{tabular}
\end{table}
\footnotetext{Maxi es no vidente, miembro activo en temas de accesibilidad en la Universidad y ha colaborado en pruebas para varios desarrollos del \gls{cespi}}

\paragraph{Selección de coordinación}
\label{desarrollo_pruebas_usabilidad_preparativos_coordinacion}

El \eng{focus group} se llevó a cabo bajo la guía de Paola Amadeo e Ivana Harari (además de mí).

\subsubsection{Etapa de diseño}
\label{desarrollo_pruebas_usabilidad_diseno}

El propósito de la prueba fue determinar el grado de aceptación y satisfacción de los usuarios con el diseño de la interfaz y la terminología elegida. Además se observó el comportamiento del grupo, y se les solicitó su opinión e ideas para determinar nuevas mejoras. 

Para relevar esta inforamción se requirió que cada integrante intente completar con la aplicación la siguiente lista de tareas: 
\begin{enumerate}
\item Crear una cuenta de usuario.
\item Ingresar al sistema con una cuenta del sistema.
\item Ingresar al sistema con una cuenta de Facebook.
\item Suscribirse a un servicio de la universidad: Guaraní de su facultad.
\item Suscribirse a un tema: Alguna materia de interés.
\item Revisar las últimas novedades.
\item Revisar las notificaciones.
\item Revisar el calendario visualizando por día, el día siguiente.
\item Revisar el calendario visualizando por semana, la semana siguiente.
\item Revisar el calendario visualizando por mes, el mes siguiente.
\item Guardar un evento en el calendario del teléfono.
\item Visualizar la ubicación de los comedores de la Universidad en el mapa.
\item Visualizar el recorrido de la línea universitaria en el mapa.
\item Visualizar otra información del servicio de la universidad al cual está suscrito.
\item Cerrar la aplicación: al recibir una notificación, acceder a su contenido.
\item Con la aplicación abierta: al recibir una notificación, acceder a su contenido.
\item Desuscribirse a un servicio de la universidad (puede suscribirse a uno para luego poder cancelarlo).
\item Desuscribirse a un tema (puede suscribirse a uno para luego poder cancelarlo).
\item Desloguearse de la aplicación.
\end{enumerate}

Por último se preguntó a cada persona si el diseño le resultaba agradable y claro y si era fácil y cómodo de utilizar.

\subsubsection{Desarrollo}
\label{desarrollo_pruebas_usabilidad_desarrollo}

Se realizaron dos encuentros, cada uno tuvo una duración aproximada de una hora y media. En el segundo se incorporaron algunas mejoras que surgieron del primero.
A cada participante se le indicó de dónde descargar la aplicación y cómo configurar el modo de desarrollador para poder instalarla.
Se requirió que todos utilicen Android 4.4 o superior.

Durante la segunda reunión surgieron algunos inconvenientes por un problema en la configuración del servidor respecto de las notificaciones \eng{push}. Igualmente se completó la prueba haciendo las salvedades necesarias.

\figura{03-capitulo-3/focus_group.png}{Algunas fotos del \eng{focus group}}{focus_group}{0.9}

\subsubsection{Resultados}
\label{desarrollo_pruebas_usabilidad_resultados}

Para medir los resultados, en primera instancia, se analizó el comportamiento de los participantes al resolver cada una de las tareas. En base a ello se evaluó cuánto les costó llegar a completarlas. Para esta valoración se utilizó un puntaje de 1 a 5, siendo 1 \comillas{con mucha dificulad} y 5 \comillas{sin problemas}.

En segunda instancia se habló con cada uno de los integrantes de manera individual y grupal para conocer sus opiniones.

Los resultados de la valuación junto al promedio por tarea se muestran en el cuadro \ref{resultados_focus_group}.

\begin{table}[h]
\caption{Resultados de \eng{focus group} por usuario y tarea (con promedio)}
\label{resultados_focus_group}
\begin{center}
\begin{tabular}{|c|c|c|c|c|c|c|}
\hline
 & \textbf{U1} & \textbf{U2} & \textbf{U3} & \textbf{U4} & \textbf{U5} & \textbf{Promedio} \\ \hline
\textbf{Tarea 1} & 4 & 4 & 5 & 5 & 5 & \textbf{4,6} \\ \hline
\textbf{Tarea 2} & 5 & 5 & 5 & 5 & 5 & \textbf{5} \\ \hline
\textbf{Tarea 3} & N/A & N/A\footnotemark{} & N/A & 5 & 5 & \textbf{5} \\ \hline
\textbf{Tarea 4} & 3 & N/A & 4 & 4 & 3 & \textbf{3,5} \\ \hline
\textbf{Tarea 5} & 4 & 4 & 5 & 5 & 5 & \textbf{4,6} \\ \hline
\textbf{Tarea 6} & 5 & 5 & 5 & 5 & 5 & \textbf{5} \\ \hline
\textbf{Tarea 7} & 5 & 5 & 5 & 5 & 5 & \textbf{5} \\ \hline
\textbf{Tarea 8} & 4 & N/A & 5 & 4 & 5 & \textbf{4,5} \\ \hline
\textbf{Tarea 9} & 4 & N/A & 5 & 5 & 5 & \textbf{4,75} \\ \hline
\textbf{Tarea 10} & 5 & N/A & 5 & 5 & 5 & \textbf{5} \\ \hline
\textbf{Tarea 11} & 4 & N/A & 5 & 4 & 5 & \textbf{4,5} \\ \hline
\textbf{Tarea 12} & 4 & N/A & 5 & 5 & 5 & \textbf{4,75} \\ \hline
\textbf{Tarea 13} & 5 & N/A & 5 & 5 & 5 & \textbf{5} \\ \hline
\textbf{Tarea 14} & 5 & 5 & 5 & 5 & 5 & \textbf{5} \\ \hline
\textbf{Tarea 15} & 4 & N/A & N/A & 5 & N/A & \textbf{4,5} \\ \hline
\textbf{Tarea 16} & 5 & N/A & N/A & 5 & N/A & \textbf{5} \\ \hline
\textbf{Tarea 17} & 3 & N/A & 4 & 3 & 4 & \textbf{3,5} \\ \hline
\textbf{Tarea 18} & 4 & 5 & 5 & 5 & 5 & \textbf{4,8} \\ \hline
\textbf{Tarea 19} & 5 & 5 & 5 & 5 & 5 & \textbf{5} \\ \hline
\end{tabular}
\end{center}
\end{table}

\footnotetext{Algunas tareas no llegaron a ser probadas porque: antes debían solucionarse problemas de accesibilidad, o no se llegó con el tiempo por problemas en la configuración del servidor surgidos al momento de la prueba (fallo en las notificaciones y filtros en la de red de desarrollo).}

En base a lo expuesto y lo hablado con los usuarios, se determinaron las siguientes mejoras en el diseño\footnote{Si bien todas son tenidas en cuenta, solo algunas se implementan dentro del alcance de esta tesis}:
\begin{itemize}
\item Agregar aclaración del tamaño de clave al registrar un nuevo usuario. 
\item Resaltar el proceso inicial de suscripciones (al iniciar por primera vez la aplicación).
\item Establecer ayudas para aclarar qué realiza cada botón.
\item En Android: modificar el gesto para eliminar un servicio. No es con \comillas{swipe} sino manteniendo apretada la opción.
\item Agregar un mensaje de bienvenida al abrir la aplicación mostrando un resumen de sus funcionalidades. Considerar la posibilidad de volver a reproducir y cancelar.
\item En cuanto a la accesibilidad:
\begin{itemize}
\item Completar información extra de los campos en el formulario de registro ya que el lector de pantalla solo muestra e-mail.
\item Mejorar los listados vacíos: el lector de pantalla no indica cuando el resultado de un listado es vacío.
\item Agregar indicación en el botón de borrar texto para el campo de búsqueda. 
\end{itemize}
\item Mostrar un mensaje al agregar un nuevo servicio (que requiere autenticación), indicando que las credenciales son las del servicio externo. Además es conveniente dar un indicador visual que la pantalla de  \eng{login} está fuera del contexto de la aplicación.
\item Cambiar el término de \comillas{servicio} por \comillas{aplicaciones}.
\item Agregar la posibilidad de ver un resumen de suscripciones a temas.
\item Cambiar término de \comillas{suscripción a temas} a \comillas{suscripcion a categorías}.
\end{itemize}

Además de las mejoras, los resultados fueron satisfactorios ya que los usuarios comentaron que la interfaz les resultó cómoda, agradable y fácil de utilizar.
