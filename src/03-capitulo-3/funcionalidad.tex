\subsection{Licencia}
\label{licencia}

Como se indica al comienzo del capítulo \ref{cap2}, una de las necesidades es la posibilidad de indicar que el software desarrollado sea libre para ser usado, copiado o modificado. Es por ello que se elige la licencia \gls{gnugpl} versión 3 para que aplique a este proyecto. Para llevar a cabo esto, se eligió como plataforma de distribución del código fuente a \gls{github} (utilizando \textit{git} como sistema de control de versiones).

Dentro de esta plataforma, el código de \nombreApp{} está dividido en dos:
\begin{itemize}
\item El \eng{backend}, con el desarrollo de la \gls{api} \gls{restful} en \textit{Lumen} (\gls{php}) ubicado en \url{https://github.com/tanoinc/mi-universidad-api}.
\item El \eng{frontend} con el código de la aplicación móvil para \textit{Ionic} en \url{https://github.com/tanoinc/mi-universidad-app}. Cabe destacar que el proyecto general está en la rama \textit{master}, mientras que el personalizado con la estética de la \unlp{} extiende a este y se encuentra en la rama \textit{unlp}.
\end{itemize}

Por otra parte, bajo esta licencia también es público el \eng{plugin} de \textit{Moodle} en \url{https://github.com/tanoinc/moodle-message_miuniversidad}\footnote{El nombre del repositorio es distitno respetando la nomenclatura de \eng{plugins} de \textit{Moodle}}.
Respecto a la personalización para \gls{siu} Guaraní (en \gls{chulupi}), quedará disponible para la comunidad de Universidades Nacionales, no pudiendo ser totalmente pública, debido a limitaciones en la licencia de \textit{Guaraní}.

Por último, este informe documental realizado con \textit{LaTeX} es público y está disponible en \url{https://github.com/tanoinc/tesis}. Queda libre como referencia para otras personas interesadas en el proyecto, o simplemente como modelo para la escritura del informe final en \textit{LaTeX}\footnote{El proyecto de \textit{LaTeX} creado por Cuesta Luengo y Carbone para su tesis, me ha servido como base para este informe. Disponible en \url{https://github.com/ncuesta/tesis}}.

\subsection{Funcionalidad}
\label{funcionalidad}

En este apartado se describe el funcionamiento del sistema: se revisan las distintas características que posee y la interacción entre sus distintos componentes.
Antes de continuar, se deben definir algunos conceptos que serán mencionados en varias ocasiones. 

En primer lugar, a cada una de las características que tendrá \nombreApp{} se las refiere como \textit{puntos de integración} y a estos, se los definen como: una funcionalidad clave, genérica y transversal a cualquier servicio, que tiene el potencial de ser integrada en la aplicación.
Cabe destacar que hay una relación directa entre estos puntos y la definición de la \gls{api}, ya que estos serán expuestos a través de ella. 

Por otro lado, se le llama \textit{servicios externos} (o servicios) a aquellos sistemas independientes (ajenos al proyecto) que tienen la capacidad de integrarse (o ya están integrados) a \nombreApp{}. Su comunicación se establece a través de la \gls{api} haciendo referencia a los \textit{puntos de integración}. Además, los usuarios tienen la posibilidad de añadir a sus cuentas personales, los servicios a los cuales estén interesados. 

Con respecto a los usuarios, se indica como \textit{identificador externo} (o \textit{id externo}) al valor clave que identifica unívocamente al usuario dentro del servicio externo. \nombreApp{} provee de mecanismos para relacionar el usuario de la aplicación con el proveniente del servicio (en caso que se requiera).

Por último, se denomina \textit{contextos} a las temáticas definidas por los \textit{servicios externos}. Estas permiten ser suscritas por los usuarios independientemente de si han añadido el servicio, definiendo un nivel más fino de granularidad. La suscripción a \textit{contextos} es la manera que tiene un usuario de indicar su interés por una temática y recibir toda la información relacionada a ella.
Los \textit{contextos} existen en el marco de un \textit{servicio} y su semántica es dada de acuerdo a la lógica de cada sistema. Dentro de la aplicación se los denomina \comillas{tema} o \comillas{temática}.

A continuación, el informe comienza por definir los \textit{puntos de integración}.

\subsubsection{Novedades}
\label{funcionalidad_noticias}

Una de las características a implementar (referidas en la sección \ref{aplicaciones_utiles_existentes_novedades}) es la publicación y notificación de \textit{Novedades} (también referidas como noticias). Se detecta como factor común, que los sistemas y aplicaciones móviles analizadas, tienen algún componente de \textit{novedades} y sus atributos se pueden modelar de forma genérica. Se trata de uno de los principales \textit{puntos de integración}.

Dentro de la Aplicación móvil, se destaca a las noticias como el ítem con mayor relevancia. Es la primer pantalla en aparecer al abrir la aplicación (estando el usuario \textit{logueado}), ubicada en la pestaña \comillas{Inicio}.
Su modalidad de navegación secundaria es de \textit{lista} y \textit{tarjeta} (similar a aplicaciones como \textit{Facebook}, \textit{Instagram} y \textit{YouTube}). 

\figura{03-capitulo-3/app_newsfeed.png}{Pantalla de inicio de \nombreApp{} (\eng{newsfeed}) }{app_newsfeed}{0.4}

El área donde se muestran las novedades se denomina \textit{newsfeed}. Cada una de ellas indica el título, fecha y contenido de la novedad, el servicio que la emite, el contexto (si existiera) y el alcance (global, de contexto y/o personal). Las noticias se muestran intercaladas una a continuación de la otra, de cualquier servicio y en orden por fecha descendente, siendo las más recientes, las que primero aparecen.

El alcance es representado por íconos dentro de la noticia:
\begin{itemize}
\item Un \comillas{mundo} representa que la novedad es global y puede ser vista por todos los que hayan añadido el servicio que la emite. Este tipo de noticias pueden ser útiles cuando desde el servicio externo se quiera notificar a todos sus usuarios. Por ejemplo, desde \textit{Guaraní} una facultad podría notificar a todos sus alumnos que cierto día hay asueto.
\item Una \comillas{etiqueta} indica que la noticia es enviada en el marco de un contexto. Esta novedad le aparecerá a los usuarios que estén suscritos a este. Por ejemplo, desde un sistema de aulas virtuales, en donde para su lógica, un contexto puede ser una materia, se podría enviar a todos los interesados en esa materia, el aviso de la publicación de una práctica. 
\item Si no aparece ningún ícono, indica que el mensaje es personal: sólo lo verá el usuario al que esté dirigido.
\end{itemize}

\subsubsection{Calendario}
\label{funcionalidad_calendario}

El siguiente \textit{punto de integración} aborda la planificación estudiantil por lo que implementa el calendario y los eventos. Al igual que las novedades, se destaca su importancia, es por ello que ocupan un lugar relevante dentro de la aplicación. 

\nombreApp{} permite a los \textit{servicios externos} comunicar eventos a sus usuarios.

Dentro de la aplicación móvil, en la pantalla principal, una de las tres pestañas que se muestran es la de \textit{calendario}. Al acceder a ella, su vista ofrece tres modalidades de visualización: por mes, por semana y por día (navegación primaria y secundaria con \textit{pestañas}). 

\figura{03-capitulo-3/app_calendario.png}{Calendario: en la primera se ve la vista por mes y en la segunda por día, luego de haber hecho un toque sobre el evento que figura en pantalla.}{app_calendario}{0.8}

De esta manera se permiten recorrer las fechas cercanas a la actual, dando saltos por distintos criterios de acuerdo a la amplitud temporal considerada. 

En la parte inferior de la pantalla figura el listado de eventos del día seleccionado (en el caso de la vista por mes), o sobre el calendario (para las vistas por semana y día). Al hacer un toque sobre el evento, la aplicación ofrece la opción de guardarlo en el dispositivo. Esto permite la integración con otras herramientas, como por ejemplo el calendario de Google, ayudando a la planificación del usuario.

Sus propiedades son: nombre, descripción, fecha, duración y ubicación. Además, al igual que las novedades, los \textit{servicios externos} pueden optar por enviar una notificación a los usuarios en la creación de un evento. Estos también pueden estar asociados a un contexto, ser globales o personales, regulando de esta manera su visibilidad y alcance.

\subsubsection{Contenidos}
\label{funcionalidad_contenidos}

El último \textit{punto de integración} remite a los contenidos. Estos son globales a todo el servicio y figurarán solamente cuando este sea añadido por el usuario a la aplicación. En dichas condiciones, los contenidos aparecerán en el menú lateral izquierdo (ver \eng{Side drawer} en la sección \ref{navegacion_transitoria}).

Los contenidos son recursos genéricos ofrecidos por la aplicación que representan cierto tipo de información para el usuario. Se caracterizan por:
\begin{itemize}
\item Ser generales a un servicio: ya que pertenecen a estos, son públicos (dentro del servicio) y están siempre disponibles.
\item Ser pasivos, al no iniciar la interacción con el usuario sino que este debe acceder a ellos para consultarlos.
\item Tener un rol secundario frente a los otros \textit{punto de integración}, es por ello que se separan en un menú aparte. Los contenidos no deberían tener la importancia como para figurar entre las pestañas principales\footnote{Podría ser que de acuerdo con el comportamiento de los usuarios, la importancia varíe, llevando a tener que cambiar la representación del contenido, por ejemplo, moviéndolo a una pestaña.}.
\end{itemize}

\figura{03-capitulo-3/app_contenidos_suscripciones.png}{Contenidos y suscripciones: Se ve la opción de suscripción y debajo contenidos (de prueba) asociados al servicio de la UNLP. Este contiene dos mapas y un texto.}{app_contenidos_suscripciones}{0.4}

Los \textit{servicios externos} tienen la posibilidad de crear los contenidos que deseen. Para el alcance de esta tesis se definen dos tipos de contenidos: mapas (de Google) y texto. A continuación se describen.

\paragraph{Mapa de Google}
\label{funcionalidad_contenidos_mapa} 

Este contenido muestra un mapa (de \textit{Google Maps}) sobre el cual dibujar información. Al abrirse el mapa, este realiza una solicitud \gls{http} a una \gls{url} externa para obtener datos de:
\begin{itemize}
\item Marcadores: Representan puntos en el mapa asociados con un ícono y un texto.
\item Centro: Indica las coordenadas geográficas en donde centrar la vista del mapa.
\item Polígonos y líneas: para ser trazados sobre el mapa, pudiendo indicar caminos o áreas.
\end{itemize}

Este mecanismo permite de generar mapas dinámicos en base a una solicitud externa\footnote{El formato de estos datos se especifica en la documentación}. Opcionalmente (actualmente no implementado) se proveerán mecanismos para cachear la información de los mapas.

\figura{03-capitulo-3/app_mapa.png}{Contenidos Mapa de Google: Ejemplo de dibujo del recorrido del colectivo Universitario. Se utilizan polígonos y marcadores.}{app_mapa}{0.4}

Una funcionalidad interesante es que permite enviar información del usuario que accede al mapa: Cuando este lo abre, se envían los datos de \textins{id externo} y posición geográfica a la \gls{url} especificada por el servicio. Este la procesa y genera los datos de marcadores, centro y polígonos. De esta manera se pueden realizar operaciones en base a la posición de la persona que consulta el mapa.

Por otra parte, la \gls{api} provee métodos para consultar las posiciones geográficas de otros usuarios. Todo ello se puede integrar para generar un mapa dinámico y personalizado al usuario.

\paragraph{Texto}
\label{funcionalidad_contenidos_texto} 

Se tratan de contenidos simples, muestran un texto con formato (encabezados, párrafos, listas, tablas, etcétera) con información estática similar a la de un documento. 
La \gls{api} permite recibir el contenido del texto en formato \gls{markdown} (texto plano). Este es mostrado y renderizado cuando el usuario consulta por esta opción en el menú.

Es útil para mostrar contenidos informativos, relativamente estáticos, que contienen texto.

\subsubsection{Notificaciones}
\label{funcionalidad_notificaciones}

Las notificaciones refieren a la acciones que realiza el dispositivo móvil para llamar la atención del usuario ante la ocurrencia de un nuevo evento. En particular para la aplicación móvil \nombreApp{}, se utilizan notificaciones \eng{push} y estas se dan con nuevas noticias o elementos en el calendario.

Si bien se destaca a esta característica como una funcionalidad importante, no se considera un \textit{punto de integración} ya que no tiene entidad por sí sola y siempre depende de otros \textit{puntos}, como las novedades o el calendario.

El comportamiento de las notificaciones se define común a todos los \textit{punto de integración} \comillas{notificables}.
De acuerdo a su alcance y a sus destinatarios, estas se comportan:
\begin{itemize}
\item Globales
\begin{itemize}
\item Con usuarios asociados: Notifican a todos los usuarios que utilizan los \textit{servicios externos} (se ignora listado de usuarios específicos ya que la noticia es global).
\item Sin usuarios asociados: Notifican a todos los usuarios que utilizan los \textit{servicios externos}.
\item Con contexto asociado: Notifican a todos los usuarios que utilizan los \textit{servicios externos} y a los interesados en el contexto.
\end{itemize}
\item No globales
\begin{itemize}
\item Con destinatarios: Notifican solamente a los usuarios del listado asociado.
\item Sin destinatarios: No notifican a nadie en particular. Este caso no debería existir, puesto que no tiene sentido que un ítem no tenga destinatarios, no sea global y no tenga asociado un contexto. No podría ser visto por nadie.
\item Con contexto asociado: Notifican a todos los usuarios interesados en el contexto. Además puede tener o no destinatarios particulares.
\end{itemize}
\end{itemize}

Puesto que el objetivo de enviar una notificación es el de destacar un cierto contenido, se considera importante que el usuario pueda llevar un seguimiento de lectura de los elementos notificados y resaltarlos por sobre otros de menor relevancia. Es por ello que en la pantalla principal de la aplicación, se muestra una pestaña específica con el listado de notificaciones referenciando a los \textit{puntos} \comillas{notificables}.

\figura{03-capitulo-3/app_notificaciones.png}{Notificaciones: Pantalla de notificaciones del usuario }{app_notificaciones}{0.4}

La pantalla de notificaciones, tiene una navegación secundaria de tipo \textit{lista}. Al hacer un toque sobre un ítem, este se marca como leído (tornándose gris) y se accede al detalle del mismo.
El ícono indica el tipo de notificación, siendo un \textit{símbolo de información} para novedades y un dibujo de un calendario para eventos.

\subsubsection{Otras}
\label{funcionalidad_otras}
