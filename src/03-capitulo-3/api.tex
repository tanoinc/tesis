\subsection{API}
\label{funcionalidad_noticias_api}

En la siguiente sección se describen brevemente algunos aspectos de la \gls{api} \gls{restful}. La intención no es realizar una especificación completa de toda la interfaz, sino revisar el mecanismo general de su utilización. Para la especificación completa se utiliza \textit{Swagger} y está disponible en \url{https://app.swaggerhub.com/apis/tanoinc/mi-universidad/1.0.0}.

En base a lo establecido con \gls{rest} en el marco teórico (sección \ref{caracteristicas_api_restful}), el llamado a la \gls{api} está orientado a recursos. En el caso de \nombreApp{} son los \textit{puntos de integración} y otra información. Los \comillas{verbos} que representan a las acciones sobre estos, son los métodos \gls{http}. También se indica que las solicitudes corresponden a la versión 1 de la \gls{api}, por lo que esta información va en la \gls{uri}. En base esto se determina la siguiente interfaz:

\begin{itemize}
\item Para novedades:
\begin{itemize}
\item POST /api/v1/newsfeed: Crea una novedad
\end{itemize}
\item Para eventos del calendario:
\begin{itemize}
\item POST /api/v1/calendar\_event: Crea un evento de calendario
\end{itemize}
\item Para contenidos:
\begin{itemize}
\item POST /api/v1/content/google\_map: Crea un mapa de Google
\item POST /api/v1/content/text: Crea un contenido de texto
\item DELETE /api/v1/content/{id}: Elimina un contenido
\end{itemize}
\item Para geolocalización:
\begin{itemize}
\item GET /api/v1/geolocation/user/\{id\_usuario\}: Obtiene la ubicación geográfica de un usuario
\item POST /api/v1/geolocation/users: Obtiene la ubicación geográfica de una lista de usuarios. Se utiliza el método post para poder pasar una lista de valores que puede ser extensa. Por ello requiere que la solicitud tenga un cuerpo, no pudiéndose realizar esto con el método GET.
\end{itemize}
\end{itemize}

Las solicitudes no mantienen ningún estado entre sucesivas peticiones y la autenticación (explicada en la siguiente sección) se realiza en cada \eng{request}. La respuesta es devuelta con su código en formato \gls{json} debido a las ventajas mencionadas en \ref{rest}.

En base a lo analizado en la sección \ref{autorizacion_autenticacion}, la autenticación con los dispositivos móviles se realiza con OAuth2, y los servicios externos utilizan su API key y firma (generada con la clave secreta). La figura \ref{fig:mi_universidad_api_diagrama} muestra un diagrama simplificado de la interacción y conexión de los \textit{servicios externos} y los dispositivos móviles, con la \gls{api} de \nombreApp{}. 

\figura{03-capitulo-3/mi_universidad_api_diagrama.png}{Diagrama de componentes de la API de \nombreApp{}}{mi_universidad_api_diagrama}{1}

En el siguiente ejemplo se ve un \eng{request} \gls{http} para la creación de una novedad. Esto sería lo que envía un servicio externo a \nombreApp{}.
\begin{itemize}
\item \textbf{POST /api/v1/newsfeed}\footnote{Esto es solo demostrativo. La especificación completa en Swagger puede consultarse en \url{https://app.swaggerhub.com/apis/tanoinc/mi-universidad/1.0.0}}.
\begin{itemize}
\item El cuerpo de la solicitud \gls{http} está armado para enviar una noticia con notificación \eng{push} para los usuarios con \textit{id} \comillas{id\_externo\_1}, \comillas{id\_externo\_2} y \comillas{id\_externo\_3}:
\begingroup
  \jsonfile{src/codigo/03-capitulo-3/post_newsfeed_solicitud.json}
  \captionof{listing}{Ejemplo de cuerpo JSON en solicitud de POST /newsfeed}.\label{codigo_post_newsfeed_solicitud}
\endgroup

\item La respuesta obtenida, considerando que todo funcionó correctamente, sería (con código \gls{http} 200):
\begingroup
  \jsonfile{src/codigo/03-capitulo-3/post_newsfeed_respuesta.json}
  \captionof{listing}{Ejemplo de respuesta JSON de POST /newsfeed}.\label{codigo_post_newsfeed_respuesta}
\endgroup
\end{itemize}
\end{itemize}

Otros posibles códigos \gls{http} de respuesta, dependiendo del resultado de la transacción, son:
\begin{itemize}
\item 200:
Procesado correctamente
\item 401:
No autorizado
\item 403:
Acceso prohibido
\item 422:
Error de validación
\item 500:
Errores en el servidor. Ver detalle en respuesta
\end{itemize}

En la siguiente sección se detalla la autenticación necesaria para el llamado a estos servicios desde sistemas externos. Luego especifica este aspecto, pero para los clientes móviles.

\subsubsection{Autenticación de servicios externos}
\label{autenticacion_servicios_externos}

Para poder integrar a los servicios externos en la aplicación, es necesario identificarlos, asegurarse que son quienes dicen ser y que su mensaje es el que han querido enviar. Para ello se utiliza el mecanismo visto en la sección \ref{apikey}. Se verá su implementación aplicada a este proyecto.

En primera instancia, la \gls{api} de \nombreApp{} correrá sobre \gls{https}. Esto evita que el mensaje pueda ser visto por terceros, y el ataque \gls{replay attack}.

En segundo lugar, el \eng{backend} conoce la identificación y clave secreta de todos sus servicios. Excede al alcance de esta tesis el mecanismo de envío de claves secretas a los administradores de los servicios externos. Estos deberán mantenerla guardada y asegurarse que no se revele. De todas formas, si alguna vez se ve comprometida, existe la posibilidad de que sea cambiada.

El servicio que envíe una solicitud a \nombreApp{} deberá firmar el pedido con su clave secreta (utilizando HMAC con sha256) y adjuntar el encabezado \gls{http} \lstinline{Authorizarion} con el valor \lstinline{APIKEY}, su identificación de servicio y la firma (separados por \comillas{dos puntos}). Por ejemplo, para enviar una nueva noticia, si:
\begin{itemize}
\item La API key es: \lstinline{0b0eedd9e23e30840fed24b8d1ab11c03913beca}

\item La clave secreta de API es: \lstinline{402082e593fee526fa64e13b47841fdbcc7a786d}

\item La \gls{url} es: \lstinline{https://localhost:8800/api/v1/newsfeed}

\item El método es: \lstinline{POST}

\item El contenido es (tipo application/x-www-form-urlencoded): \lstinline{title=Titulo de prueba&content=Texto contenido&send_notification=1&clobal=0&context=matematica}

\end{itemize}

La firma con HMAC (sha256) se deberá generar con la clave secreta (\lstinline{402082e593fee526fa64e13b47841fdbcc7a786d}) y el siguiente contenido\footnote{Para favorecer la visualización en este documento, el JSON se muestra formateado con saltos de línea y \textit{tabs}, pero en la implementación estos no existen. Esto es importante ya que una diferencia en un caracter, cambia la firma.}:

\begingroup
  \jsonfile{src/codigo/03-capitulo-3/nuevo_newsfeed.json}
  \captionof{listing}{JSON generado en base a una solicitud utilizando los atributos del \eng{request} HTTP.}
\endgroup

El resultado es la siguiente firma:

\lstinline{a42e995848c48f51a226b5089196d225e0635d3536be126c0331a9cde700d398}


Notar que todos los datos de la solicitud se utilizan en para generar el \eng{hash} con la firma, por lo que si alguno cambia, esta firma ya no tendría sentido.

La solicitud \gls{http} enviada quedaría de la siguiente manera\footnote{La API key y la firma se recortan para que puedan entrar en el ancho de la página}:
\begingroup
  \httpfile{src/codigo/03-capitulo-3/nuevo_newsfeed.http}
  \captionof{listing}{JSON generado en base a una solicitud utilizando los atributos del \eng{request} HTTP.}
\endgroup

De esta manera el servidor realiza el proceso inverso y si las firmas son iguales, entonces la solicitud es válida: es de quien dice ser y no fue alterada.

En la figura \ref{fig:mi_universidad_api_diagrama} este mecanismo se realiza entre los servicios externos y la interfaz con \textit{API key y secret}.

\paragraph{Permisos}
\label{autenticacion_permisos}

Los permisos indican qué operaciones pueden ser utilizadas por cada servicio. Existe un permiso por cada entrada de la \gls{api} (\gls{url} y método \gls{http}).

La aplicación \nombreApp{} maneja un sistema que controla qué operaciones están habilitadas para cada servicio. Esta asociación permiso-servicio está indicada con un número de versión que luego es utilizado cuando un usuario añade un servicio. Este es aceptado en una versión, por lo que si se modifica deberá volver a aceptarse. De esta manera se informa al usuario de cada cambio en las operaciones que requieran los servicios.

\subsubsection{Autenticación con aplicación móvil}
\label{autenticacion_con_app}

La autenticación con el cliente móvil se realiza mediante OAuth2 (según lo definido en el marco teórico de la sección \ref{autorizacion_autenticacion}) con el tipo de accedo \textit{password grant}. De su utilización se destacan cuatro pasos:
\begin{itemize}
\item La solicitud del token de acceso: mediante un \textit{POST} a la \gls{api} en \textit{/oauth/token} con las credenciales. Esta devuelve un token de acceso y refresco en formato \gls{jwt}, que serán utilizados en todos los request subsiguientes como mecanismo de autenticación.
\item La utilización del token en los llamados: En cada solicitud para el acceso a un recurso, se debe enviar en el encabezado \gls{http} \lstinline{Authorization}, el tipo de token (en este caso \textit{Bearer}) y el token de acceso.
\item La actualización del token vencido: Cuando pierda validez el token de acceso, se utiliza el token de refresco para solicitar uno nuevo de acceso.
\item Salir del sistema: Se envía una solicitud \textit{DELETE} con el token y se invalidan los tokens evitando ser utilizados en futuros accesos.
\end{itemize}

En la figura \ref{fig:mi_universidad_api_diagrama} este mecanismo se refleja entre los clientes móviles y la interfaz con Oauth2.