\subsection{Análisis sobre los Sistemas Operativos Móviles}
\label{sistemas_operativos}

En el mercado actual existe una amplia variedad de empresas que venden
teléfonos móviles inteligentes.

Desde el punto de vista del desarrollo, el 80\% de aquellos que desarrollan
aplicaciones móviles profesionalmente, apuntan a Android como su plataforma
primaria. Con el 53\% iOS y 30\% el navegador del móvil\cite{DevNation2}.

\figura{01-capitulo-1/mobile-platform.jpg}{Porcentaje de
desarrolladores que elijen como plataforma primaria el
sistema operativo.}{dev_nation_so}{1}

Particularmente en sudamérica, el porcentaje de desarrolladores móviles
priorizando la plataforma Android es del 38\% de manera profesional profesional,
más 18\% como hobby o proyecto secundario. En segundo lugar está iOS con el 15\%
profesional y 3\% como hobby o proyecto secundario\cite{DevNation2}.

Desde el punto de vista del uso, la venta de teléfonos celulares en todo el
mundo con Android como sistema operativo representó el 86,2\% del total de las
ventas, frente al 12,9\% de iOS y 0,9\% entre Windows, BlackBerry y otros.
En Argentina Android representa el 79\% de los sistemas operativos del mercado,
Windows 12\% e iOS, 3,5\%\cite{SmartphoneMarketShare}.
