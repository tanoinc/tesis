\subsection{Análisis sobre los Sistemas Operativos Móviles}
\label{sistemas_operativos}

En el mercado actual existe una amplia variedad de empresas que venden
teléfonos móviles inteligentes.

Desde el punto de vista del desarrollo, el 80\% de aquellos que desarrollan
aplicaciones móviles profesionalmente, apuntan a Android como su plataforma
primaria. Con el 53\% iOS y 30\% el navegador del móvil\cite{DevNation2}.

\figura{01-capitulo-1/mobile-platform.jpg}{Porcentaje de
desarrolladores que elijen como plataforma primaria el
sistema operativo.}{dev_nation_so}{1}

Particularmente en sudamérica, el porcentaje de desarrolladores móviles
priorizando la plataforma Android es del 38\% de manera profesional profesional,
más 18\% como hobby o proyecto secundario. En segundo lugar está iOS con el 15\%
profesional y 3\% como hobby o proyecto secundario\cite{DevNation2}.

Desde el punto de vista del uso, la venta de teléfonos celulares en todo el
mundo con Android como sistema operativo representó el 86,2\% del total de las
ventas, frente al 12,9\% de iOS y 0,9\% entre Windows, BlackBerry y otros.
En Argentina Android representa el 79\% de los sistemas operativos del mercado,
Windows 12\% e iOS, 3,5\%\cite{SmartphoneMarketShare}.

De estos hechos se desprende la necesidad de considerar un desarrollo que permita ser implementado en múltiples plataformas móviles. Se tendrá en cuenta para el alcance de esta tesis, Android, puesto que es el principal sistema operativo móvil del mercado en Argentina\cite{SmartphoneMarketShare}, además de su importante adopción en el mundo del desarrollo\cite{DevNation2}.

\subsubsection{Android: Versiones utilizadas en la UNLP}
\label{sistemas_operativos_versiones}

 Se analizaron los datos captados mediante la utilidad \eng{Google Analytics} sobre los usuarios del sistema \gls{guarani} en la \unlp{}\footnote{Estos datos fueron obtenidos desde el \gls{cespi}, lugar en donde trabajo.}. Esta herramienta permite discriminar, en los accesos a su interfaz Web, por Sistema Operativo y versión. En particular para Android, la información extraída sirve como un indicador de las versiones que tienen instaladas los estudiantes.

Los datos se obtuvieron desde el 1 de Enero de 2017 hasta el 6 de Septiembre de 2017, y comprenden las \comillas{fechas pico} del uso del sistema: las inscripciones a cursadas. Se analizaron 22 instalaciones correspondientes a todas las Facultades de la \unlp{} y otras dependencias (como postgrados).
Las cantidades cuentan lo que Google denomina como \textit{sesiones}: \comillas{una sesión es el elemento que engloba las acciones del usuario en su sitio web}, y como \textit{acciones} a \comillas{un conjunto de interacciones que tienen lugar en su sitio web en un periodo determinado}\cite{google2017analyticsSesion}.
Cabe destacar que bajo estos conceptos, un usuario podría ingresar dos veces al sitio en diferentes momentos y se contaría como dos sesiones\footnote{Inclusive las sesiones pueden ser con dos sistemas operativos distintos (o distintas versiones).}. 

En el cuadro \ref{android_versiones_cantidad} se pueden ver las cantidades por versión de Android.

\begin{table}[htbp]
\centering
\caption{Cantidad de sesiones en Guaraní desde Android, por versión (Enero a Septiembre 2017)}
\label{android_versiones_cantidad}
\begin{tabular}{|l|r|}
\hline
\textbf{Versión de Android} & \multicolumn{1}{l|}{\textbf{Cantidad de sesiones}} \\ \hline
8   & 84     \\ \hline
7   & 75886  \\ \hline
6   & 450905 \\ \hline
5   & 417938 \\ \hline
4.4 & 165938 \\ \hline
4.3 & 12858  \\ \hline
4.2 & 17571  \\ \hline
4.1 & 27517  \\ \hline
4.0 & 4239   \\ \hline
3   & 37     \\ \hline
2   & 622    \\ \hline
1   & 1     \\ \hline
\end{tabular}
\end{table}

Del cuadro \ref{android_versiones_cantidad} se desprenden los porcentajes de la figura \ref{fig:android_porcentaje_versiones}. En el gráfico se puede ver que mas del 94\% utilizan la versión 4.4 o superiores.

\figura{01-capitulo-1/android_versiones.png}{Porcentaje de uso en versiones de Android en accesos a Guaraní Web}{android_porcentaje_versiones}{1}

En el anexo \ref{cantidad_versiones_android_unlp} se detallan los valores por Unidad Académica.