\subsection{Aplicaciones universitarias existentes}
\label{aplicaciones_existentes}

Actualmente existen varias soluciones para dispositivos móviles en el contexto universitario. Tanto aplicaciones que abarcan aspectos puntuales en el proceso de educación, como plataformas para el desarrollo de herramientas universitarias. Se analizarán los aspectos interesantes de las aplicaciones más importantes.

\subsubsection{Kurogo (Modo Labs)}
\label{aplicaciones_existentes_kurogo}

\figura{01-capitulo-1/kurogo_app.png}{Menú principal de aplicaciones realizadas
con Kurogo.}{kurogo_app}{0.5}

Kurogo es una plataforma desarrollada por Modo Labs para crear aplicaciones
móviles (con sistema operativo Android o iOS) orientadas a la favorecer la
comunicación y dar a conocer información acerca de entidades, en especial,
universitarias, de manera sencilla. Dicha plataforma está organizada en una
serie de módulos (expresados a través íconos con texto en la pantalla principal)
que representan datos y actividades de interés para el estudiante: servicios que
ofrecen, enlaces, eventos, horarios, teléfonos y mapas. Se destacan algunas de
sus funciones:
\begin{itemize}
  \item \textbf{Calendario}: Agrupados por las categorías a las que pertenecen,
  se muestra información de los eventos relacionados a la universidad, con la
  posibilidad de buscarlos, compartirlos y agregarlos al calendario del dispositivo.
  \item \textbf{Mensajería}: Le permite a los estudiantes recibir notificaciones
  push y mensajes en forma de avisos en tira\footnote{Los avisos en tira son
  mensajes breves que aparecen en la parte superior de la pantalla.}.
  \item \textbf{Bibliotecas}: Permite realizar búsquedas de libros y artículos y
  consultar su información y disponibilidad Mapa: ofrece un mapa completo de los
  edificios internos del campus universitario (de interiores y exteriores).
  Existe también la posibilidad de realizar búsquedas.
  \item \textbf{Emergencias}: Permite recibir noticias críticas de emergencias y
  acceder al listado de teléfono útiles.
  \item \textbf{Comedor}: Da a conocer los menús y platos que ofrece el comedor,
  junto con sus horarios.
  \item \textbf{Estacionamiento}: Este módulo permite recibir información en
  vivo de los lugares libres para estacionar el automóvil. Requiere de software
  y hardware extra para permitir esta funcionalidad.
\end{itemize}
Kurogo es utilizada por Universidades como Colgate University, Harvard y CSUN, entre otras.
\subsubsection{Universidad de Harvard}
\label{aplicaciones_existentes_harvard}
La Universidad de Harvard posee varias aplicaciones móviles para sus estudiantes. 
La principal, Harvard Mobile, está desarrollada utilizando Kurogo (antes
mencionada) y brinda la información básica que provee esta plataforma. Está
disponible para iOS, Android y web móvil\footnote{Sitio web optimizado para
móviles: https://m.harvard.edu/}.
También disponen de tres aplicaciones orientadas tours y recorridos virtuales
(con diferentes temáticas históricas, culturales y botánicas), y otra
relacionada con eventos y noticias de la escuela de salud pública.
