\subsection{Aplicaciones móviles útiles para el ámbito universitario}
\label{aplicaciones_utiles_existentes}

En la actualidad existen muchas aplicaciones para ayudar en cada aspecto de
nuestra vida cotidiana, y el ámbito universitario no escapa a ello. Se
revisarán las mas útiles \cite{StudentApps} para poder analizar funcionalidades
interesantes que puedan mejorar la experiencia de los estudiantes.

\subsubsection{Planificación estudiantil}
\label{aplicaciones_utiles_existentes_planificacion}

La planificación es clave para la vida universitaria de los estudiantes. Es por
ello que resulta útil una herramienta que ayude a organizar fechas y horarios
de cursada, parciales, finales y entregas en un cronograma. Existen 
aplicaciones para manejar esta inforamción.

En Android está disponible \eng{Timetable}
\footnote{Timetable disponible en Google Play: 
\url{https://play.google.com/store/apps/details?id=com.gabrielittner.timetable}}:
que permite (de manera intuitiva) administrar las tareas, exámenes, permitiendo la sincronización con otros dispositivos. Como característica novedosa, silencia el teléfono en los horarios
de clase.
Para iOS existe \eng{Class Timetable}
\footnote{Class Timetable disponible en
\url{https://itunes.apple.com/gb/app/class-timetable/id425121147?mt=8}} similar
a la anterior, pero permitiendo el envío de notificaciones de tareas o
vencimientos y mejores visualizaciones para las actividades de la semana.

\subsection{Aplicaciones universitarias existentes}
\label{aplicaciones_universitarias_existentes}

Actualmente existen varias soluciones para dispositivos móviles en el contexto
universitario. Tanto aplicaciones que abarcan aspectos puntuales en el proceso
de educación, como plataformas para el desarrollo de herramientas
universitarias. Se analizarán los aspectos interesantes de las aplicaciones más
importantes.

\subsubsection{Kurogo (Modo Labs)}
\label{aplicaciones_existentes_kurogo}

\figura{01-capitulo-1/kurogo_app.png}{Capturas de pantalla del menú principal de
aplicaciones realizadas con Kurogo.}{kurogo_app}{0.5}

Kurogo es una plataforma desarrollada por Modo Labs para crear aplicaciones
móviles (con sistema operativo Android o iOS) orientadas a la favorecer la
comunicación y dar a conocer información acerca de entidades, en especial,
universitarias, de manera sencilla. Dicha plataforma está organizada en una
serie de módulos (expresados a través íconos con texto en la pantalla principal
como se muestran en la figura \ref{fig:kurogo_app}) que representan datos y
actividades de interés para el estudiante:
servicios que ofrecen, enlaces, eventos, horarios, teléfonos y mapas. Se destacan algunas de
sus funciones:
\begin{itemize}
  \item \textbf{Calendario}: Agrupados por las categorías a las que pertenecen,
  se muestra información de los eventos relacionados a la universidad, con la
  posibilidad de buscarlos, compartirlos y agregarlos al calendario del dispositivo.
  \item \textbf{Mensajería}: Le permite a los estudiantes recibir notificaciones
  push y mensajes en forma de avisos en tira\footnote{Los avisos en tira son
  mensajes breves que aparecen en la parte superior de la pantalla.}.
  \item \textbf{Bibliotecas}: Permite realizar búsquedas de libros y artículos y
  consultar su información y disponibilidad Mapa: ofrece un mapa completo de los
  edificios internos del campus universitario (de interiores y exteriores).
  Existe también la posibilidad de realizar búsquedas.
  \item \textbf{Emergencias}: Permite recibir noticias críticas de emergencias y
  acceder al listado de teléfono útiles.
  \item \textbf{Comedor}: Da a conocer los menús y platos que ofrece el comedor,
  junto con sus horarios.
  \item \textbf{Estacionamiento}: Este módulo permite recibir información en
  vivo de los lugares libres para estacionar el automóvil. Requiere de software
  y hardware extra para permitir esta funcionalidad.
\end{itemize}
Kurogo es utilizada por Universidades como Colgate University, Harvard y CSUN, entre otras.
\subsubsection{Universidad de Harvard}
\label{aplicaciones_existentes_harvard}
La Universidad de Harvard\cite{HarvardMobile} posee varias aplicaciones
móviles para sus estudiantes.
La principal, Harvard Mobile, está desarrollada utilizando Kurogo (antes
mencionada) y brinda la información básica que provee esta plataforma. Está
disponible para iOS, Android y web móvil\footnote{Sitio web optimizado para
móviles: https://m.harvard.edu/}.
También disponen de tres aplicaciones orientadas tours y recorridos virtuales
(con diferentes temáticas históricas, culturales y botánicas), y otra
relacionada con eventos y noticias de la escuela de salud pública.

\subsubsection{Universidad de Oxford}
\label{aplicaciones_existentes_oxford}

\figura{01-capitulo-1/oxford_app.png}{Captura de pantalla de la aplicación
oficial de la universidad de Oxford.}{oxford_app}{0.3}

Para la Universidad de Oxford\cite{OxfordMobile} existe una aplicación principal
(\eng{Oxford University: The Official Guide app}, disponible solo para \gls{ios}) que está
centrada en mostrar las novedades y los recorridos turísticos focalizados en
distintas temáticas como esculturas, jardines y ganadores del premio Nobel,
entre otras. Permite también, conocer qué hay en los alrededores y mostrar
sugerencias de qué hacer en esa zona.

Existen además otras \eng{apps} secundarias que permiten ver la revista de
la institución (\eng{Oxford Today}) y recorrer sus museos (\eng{Explore Oxford
University Museums}) y un sitio web adaptado con estilos móviles en el que sus  
principales funciones son son: conocer las tareas diarias, verificar el estado
del transporte en colectivo, encontrar un libro de la biblioteca, etcétera.


