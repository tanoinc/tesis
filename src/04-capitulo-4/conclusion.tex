
Las universidades de Argentina, y específicamente la Universidad Nacional de La Plata y sus dependencias, disponen de un conjunto de sistemas variados. En base a esto, y luego de transitar laboralmente por el \gls{cespi}, afirmo desde mi experiencia que uno de los aspectos que se destacan en cada desarrollo es el de favorecer y mejorar la integración de dichos sistemas universitarios. Por ende, el desarrollo de la aplicación \nombreApp{} pretende extender este mismo lineamiento conceptual hacia los alumnos, para favorecer y facilitar su tránsito académico por la Universidad.

De lo expuesto en la presente tesina se concluye que: se puede desarrollar una solución para facilitar el acceso a la información por parte de los alumnos, que sea:
\begin{itemize}
\item abierta (utilizando software \gls{open source} y estándares abiertos),
\item distribuida,
\item extensible a otras Universidades,
\item con una interfaz pública que pueda ser consultada y utilizada por otros servicios,
\item que permita integrar aspectos claves de múltiples sistemas.
\end{itemize}
Estas características ayudan a mejorar la experiencia de usuario de múltiples sistemas universitarios, manteniendo una comunicación inmediata con la casa de altos estudios.

Lo anteriormente mencionado fue llevado a cabo mediante el estudio de las últimas tecnologías y versiones. Para el cliente (\eng{frontend}) se revisaron varios \glspl{framework} para el desarrollo de una aplicación híbrida. Se concluyó la utilización y el estudio de Ionic 2 (con Cordova y tecnologías Web como \gls{html}5, \gls{css}3, \gls{javascript} con Angular 4 mediante \gls{typescript}). Para el servidor (\eng{backend}) se utilizó el \gls{framework} \gls{php} Lumen, derivado de \gls{laravel} orientado a la creación de \glspl{api} (con autenticación OAuth2 y \glspl{api} \eng{key/secret}). Por último, para las extensiones se utilizó \gls{siu} \gls{chulupi} (Guaraní) y para Moodle se estudió el tipo de \eng{plugin} \eng{message output}.

Esto permitió la implementación de: 
\begin{itemize}
\item una \gls{api} \gls{restful} que integra los servicios externos e interactúa con el cliente móvil y, 
\item una aplicación móvil como cliente consumidor y fachada principal para los estudiantes de la Universidad en donde sus características son comunes entre las aplicaciones universitarias analizadas.
\end{itemize}

Por otro lado, debido a la posibilidad de acceso a sistemas administrados por el \gls{cespi}, se desarrollaron dos módulos para integrar con \nombreApp{}: en \gls{guarani} y en Moodle.

Finalmente, los \eng{focus group} fueron claves como primera instancia real de análisis de uso. Produjeron muchas modificaciones que favorecieron la usabilidad e hicieron que la aplicación se vaya \comillas{puliendo} en cada reunión.

Sin embargo, quedan muchas ideas por implementar y tareas por completar. Seguidamente se revisan las pendientes.

\subsection{Trabajos futuros}
\label{trabajos_futuros}

A continuación se destacan algunos trabajos futuros (muchos de ellos serán llevados a cabo en el marco de mi trabajo en el \gls{cespi}):

\begin{itemize}
\item Focalizar la implementación inicial de la aplicación a los alumnos internacionales de la \unlp{}.
\item Mejorar contenidos existentes e incorporar nuevos. Esto debe hacerse teniendo en cuenta que estos sean genéricos y no involucren en aspectos específicos de la lógica correspondiente a los servicios externos. Considerar también aumentar la integración con redes sociales, permitiendo la incorporación de novedades desde páginas de \textit{Facebook}, \textit{Twitter} y  \textit{Google+}.
\item Incorporar un motor de búsqueda (como Sphinx, ElasticSearch, Lucene, etc.) para buscar sobre novedades, calendario y todo contenido \comillas{buscable}.
\item Analizar la integración de la \gls{api} dentro de un \textit{API Gateway} (como \textit{Tyk}) para favorecer la implementación del modelo de microservicios.
\item Ampliar la compatibilidad de la aplicación con versiones de \gls{android} menores a 4.4 (mediante \eng{Crosswalk}) y compilarlas para otros Sistemas Operativos (\gls{ios} y Windows).
\end{itemize}

\subsection{Carrera, Tesis y experiencia}
\label{conclusion_carrera}

La formación académica brindada por la \unlp{} y la \facultad{}, me ha dotado de la base de conocimientos teóricos y prácticos necesarios para llevar a cabo todas las etapas del desarrollo. Los contenidos necesarios para este proyecto son transversales a gran parte (si no es que todas) de las asignaturas de la Licenciatura. Habiendo finalizado el desarrollo, recorrí año a año los temas vistos en cada una de las materias y encontré que hay algo de todas ellas en \nombreApp{}. 

Por otra parte, como complemento a la formación académica, la experiencia ganada trabajando en el \gls{cespi} fue clave. Esta no se trata solamente en la especialización de tecnologías utilizadas en desarrollos actuales, sino también como resultante del incentivo a la investigación, el facilitamiento de acceso a nuevas tecnologías y el aporte de las herramientas necesarias. 

Otro aspecto a destacar son las relaciones humanas, en donde se aprende a interactuar con otras personas: pares y de otras áreas, formando grupos interdisciplinarios. En la Facultad y el trabajo es donde ganan compañeros, colegas y amigos que lo acompañan en la formación académica, profesional y personal.

Para finalizar, el desarrollo y la investigación realizados para esta tesina me han hecho aprender el uso de \glspl{framework} actuales para el desarrollo de aplicaciones móviles híbridas, y de \glspl{pwa}, además de novedades en tecnologías Web, como las últimas versiones de Angular con \gls{typescript}. Por otra parte, he podido conocer y considerar alternativas para el desarrollo móvil, ampliar el conocimiento sobre las tecnologías de autenticación/autorización de \glspl{api} y profundizar sobre aspectos de usabilidad mediante los \eng{foucs group}. Como un último aspecto, he aprendido buscar, leer, organizar y escribir textos científicos. No sólo para la redacción de esta tesina, sino también para la investigación de artículos, publicaciones y tesis de colegas.
