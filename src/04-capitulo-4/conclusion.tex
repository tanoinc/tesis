
Las universidades de Argentina, y específicamente la Universidad Nacional de La Plata y sus dependencias, disponen de un conjunto de sistemas variados. En base a esto, y luego de transitar laboralmente por el \gls{cespi}, afirmo desde mi experiencia que uno de los aspectos que se destacan en cada desarrollo es el de favorecer y mejorar la integración de dichos sistemas universitarios. Por ende, el desarrollo de la aplicación \nombreApp{} pretende extender este mismo lineamiento conceptual hacia los alumnos, para favorecer y facilitar su tránsito académico por la Universidad.

De lo expuesto en la presente tesina se concluye que: se puede desarrollar una solución para facilitar el acceso a la información por parte de los alumnos, que sea:
\begin{itemize}
\item abierta (utilizando software \gls{open source}),
\item distribuida,
\item extensible a otras Universidades,
\item con una interfaz pública que pueda ser consultada y utilizada por otros servicios,
\item que permita integrar aspectos claves de múltiples sistemas.
\end{itemize}
Estas características ayudan a mejorar su experiencia como usuarios de múltiples sistemas universitarios, manteniendo una comunicación inmediata con la casa de altos estudios.

\subsection{Trabajos futuros}
\label{trabajos_futuros}

A continuación se destacan algunos trabajos futuros (muchos de ellos serán llevados a cabo en el marco de mi trabajo en el \gls{cespi}):

\begin{itemize}
\item Mejorar contenidos existentes e incorporar nuevos:
\begin{itemize}
\item Nuevo contenido de tipo \comillas{Certificado}: que permita a los servicios asociar certificados, pudiendo ser estos accedidos desde la aplicación, generando un texto y un código QR de validación del mismo. Este código contendría una \gls{url} apuntando a un servicio externo que controla su validez. 
\item Mejorar el contenido de \comillas{Mapas} para que permita mayor interacción de usuario, como puede ser la incorporación de algunos controles o la actualización dinámica de los datos dibujados sobre el mapa.
\item Incorporar más tipos de contenido, siempre teniendo en cuenta que estos sean genéricos y no se involucres en aspectos específicos de la lógica correspondiente a los servicios externos.
\item Implementar el contenido de mapas con \textit{OpenStreetMaps} o alternativas abiertas.
\item Enviar avisos (en novedades) al haber cambios en los contenidos.
\end{itemize}
\item Incorporar un motor de búsqueda (como Sphinx, ElasticSearch, Lucene, etc.) para buscar sobre novedades, calendario y todo contenido \comillas{buscable}.
\item Mejorar la integración con redes sociales, permitiendo la incorporación de novedades desde páginas de \textit{Facebook}, \textit{Twitter} y Google+.
\item Incorporar la posibilidad de insertar multimedia en las novedades.
\item Analizar la integración de la \gls{api} dentro de un \textit{API Gateway} (como \textit{Tyk}) para favorecer la implementación del modelo de microservicios.
\item Ampliar la compatibilidad de la aplicación con \gls{ios} y versiones de \gls{android} menores a 4.4 (mediante \eng{Crosswalk}). Comprobar que la funcionalidad sea correcta.
\end{itemize}