\subsection{Introducción}
\label{analisis_servicios_introduccion}

Este capítulo presenta el marco teórico y el análisis de las tecnologías existentes, para fundamentar el desarrollo propuesto y su implementación.

Es importante destacar que para el marco teórico del proyecto propuesto en esta tesina, surge la necesidad de considerar los mecanismos para:
\begin{itemize}
\item Brindar un mecanismo genérico para integrar información útil para el estudiante, entre distintos sistemas implementados en la \unlp{}.
\item Proveer nuevas operaciones y que estas estén disponibles para su uso externo, a través de otros sistemas.
\item Comunicar datos de interés entre el \eng{backend} y los dispositivos móviles.
\item Representar estos datos en una aplicación móvil, de manera sencilla y que el usuario encuentre cómodo al momento de su utilización.
\item Expresar la libertad de uso, copia y modificación del desarrollo.
\end{itemize}

En primera instancia se revisan las razones del uso de software libre.
En segundo lugar, puesto que el modelo de aplicación distribuida elegido es el de cliente-servidor, este apartado continúa con dos secciones. La primera, revisa las características técnicas para proveer los recursos (\eng{backend}, servidor) y la segunda, detalla sobre el consumidor (\eng{frontend}, cliente).

Por otra parte, considerará algunos de los servicios implementados actualmente por la \unlp{} (y sus dependencias) para lograr su integración. Además, se considerará la potencialidad de incorporación de otros servicios pre-existentes a este desarrollo.


