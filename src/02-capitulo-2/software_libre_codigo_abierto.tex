\subsection{Software libre y de código abierto}
\label{software_libre_codigo_abierto}

El software libre y de código abierto (denominado \gls{foss}) es aquel que reúne ambas características: la de ser \textit{Software libre} y de \textit{código abierto}. Esto significa que cualquiera es libre de usar, copiar, estudiar y cambiar el software, y además que su código es abiertamente compartido para motivar a las personas a que mejoren su diseño\cite{libertades1996fsf}.

Aunque ambos términos parezcan similares y estén motivados por cuestiones en común, existen diferencias: el \textit{código abierto} se basa en las ventajas que posee este modelo de desarrollo, mientras que el \textit{software libe} es un concepto más filosófico que trata de las libertades de los usuarios respecto de los programas.

El software libre se base en que los programas deben respetar cuatro libertades. A continuación son enumeradas, citando textualmente a la \eng{Free Software Foundation} en \cite{libertades1996fsf}:
\blockcquote{libertades1996fsf}{
La libertad de ejecutar el programa como se desea, con cualquier propósito (libertad 0).

La libertad de estudiar cómo funciona el programa, y cambiarlo para que haga lo que usted quiera (libertad 1). El acceso al código fuente es una condición necesaria para ello.

La libertad de redistribuir copias para ayudar a su prójimo (libertad 2).

La libertad de distribuir copias de sus versiones modificadas a terceros (libertad 3). Esto le permite ofrecer a toda la comunidad la oportunidad de beneficiarse de las modificaciones. El acceso al código fuente es una condición necesaria para ello.
}

Los beneficios de estos principios\cite{noyes2010openSourceGood} son:
\begin{itemize}
\item La seguridad: Cuantas más personas vean el código, es más probable que detecten errores y los corrijan. Esto tiene un impacto directo en el marco de la seguridad.
\item La calidad: En relación con el inciso anterior, la cantidad de usuarios de un desarrollo, también influye, ya que permite que estos incorporen nuevas funcionalidades o las mejoren.
\item Personalización: Al permitir modificaciones, habilita a que estas se realicen para adaptarse a las necesidades del usuario u organismo.
\item La libertad: La utilización de software de \textit{código abierto} libera el hecho de \comillas{estar atado} a una tecnología propietaria.
\item La interoperabilidad: Suele adhiere más a los estándares libres que el software privativo, lo que evita estar limitado al uso de formatos cerrados.
\item La auditabilidad: La visibilidad del código permite a los usuarios ver las acciones que este ejecuta.
\item Las opciones de soporte: El soporte es gratis a través de la asistencia de la comunidad de usuarios y desarrolladores. También existe el soporte pago, cuando es requerido asegurarse un mantenimiento.
\item La gratuidad (sin costo): por definición es gratis.
\item Las pruebas de un producto: Ayuda a evaluar un software antes de utilizarlo.
\end{itemize}

Estos principios motivan su adopción para el desarrollo de software realizado para esta tesina y como una característica requerida en los componentes que esta utilice. 

\subsubsection{GNU \eng{General Public License} versión 3}
\label{gnu_gpl_v3}

La licencia púbica general GNU (\gls{gnugpl}) fue creada por Richard Stallman para el proyecto GNU. Esta brinda garantías al usuario final para utilizar, compartir, estudiar y cambiar el software. Su objetivo es declarar que los desarrollos que estén bajo esta garantía sean libres y estén protegidos por \eng{copyleft}, evitando que futuras modificaciones por terceros restrinjan las libertades que brinda esta licencia.

En al año 2007 se publicó la versión 3. En líneas generales esta nueva versión incorporó los siguientes cambios\cite{gpl3guide2014fsf}:
\begin{itemize}
\item Neutralizar las leyes que prohíben el software libre.
\item Reforzar la protección contra amenazas de patentes.
\item Aclarar la compatibilidad de la licencia con otras.
\item Agregar licencias compatibles.
\item Habilitar nuevas maneras para compartir el código fuente.
\item Distribuir menos código fuente (agregando excepciones para librerías comunes).
\item Ajustes para hacer mas global a la licencia.
\end{itemize}

Para terminar, todo el código fuente escrito para el desarrollo de esta tesina adhiere a la licencia \gls{gnugpl} versión 3.
